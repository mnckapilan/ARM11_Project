\documentclass[11pt]{article}

\usepackage{fullpage}

\begin{document}

\title{ARM Emulator Report}
\author{
    Patrick Henderson\\
    \and
    Sukant Roy\\
    \and
  	Kapilan Manu Neethi Cholan\\
  	\and
  	Jordan Bunke
}

\maketitle

\section{Group Organisation}

	Group organisation content here.

\section{Implementation Strategies}
    I have implemented the execution of the single data transfer instruction. The
    main challenge initially was to figure out how to extract the relevant bits from
    the instruction. I decided to use bit masks, which I defined as hexadecimal
    constants in the source file. I will be able to use this approach for the assembler.
    I will also be able to use the constants corresponding to my instruction field
    indexes, as well as the constant bit masks for extracting each field in the
    instruction, all of which I defined in the source file using #define. My header
    file consists of a single function declaration, for the function which executes
    the instruction itself. None of the other functions defined in my source file
    are useful to other group members: they are all used for executing my specific
    instruction type, so are only relevant to my source file. All the constants I
    have defined for executing my instruction are local to my source file, so don't
    need to be defined in the header file. My source file has a function which
    executes the instruction, as well as helper functions used by this function,
    which perform smaller tasks, for example transferring data or computing a memory
    address. This ensures my execute function isn't too long.

\end{document}
